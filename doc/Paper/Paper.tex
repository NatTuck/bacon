\documentclass{llncs}

\begin{document}

\title{Bacon: GPU Programming With Just in Time Specialization}

\author{Nat Tuck\inst{1}}

\institute{University of Massachusetts Lowell, Lowell MA 01854, USA}

\maketitle

\begin{abstract}

This paper describes Bacon, a programming language that provides
simpler development and better optimizations for OpenCL compatible
general purpose GPUs.

\end{abstract}

\section{Introduction}

OpenCL is a standard language for data parallel computation primarily
targetted at graphics processing units.

Although OpenCL maps well to GPU hardware and provides detailed
control, it has some weaknesses. The C++ interface to OpenCL is very
verbose, and requires quite a bit of boilerplate to get any program to
run at all. OpenCL doesn't provide any mechanism for dynamic memory
allocation. Existing implementations of OpenCL don't do loop unrolling
for loops with unknown iteration counts.

Bacon was developed to try to improve the OpenCL programming
experience. It extends the OpenCL C programming language and provides
a compiler that hopes to solve these issues. C++ wrapper code is
generated automatically. Just in time specialization allows the 
illusion of dynamic stack allocation as well as the unrolling of
some loops.

Partial evaluation has been shown to provide improved performance in
scientific computing applications. A good example of this is shown in
\cite{Berlin:1990}.

The design of the OpenCL runtime system depends on device-specific
code generation occuring at runtime, which suggests the use of
parameter information for optimization at runtime.

\section{Bacon}
\subsection{Language}

The Bacon programming language is based on OpenCL C with a small
number of modifications. 

- 


\subsection{Implementation}

Hai

\section{Result}

Hai

\section{Conclusion}

Hai

\bibliography{bacon}{}
\bibliographystyle{splncs03}
\end{document}
